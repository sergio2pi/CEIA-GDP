\documentclass[
11pt % The default document font size, options: 10pt, 11pt, 12pt
%codirector, % Uncomment to add a codirector to the title page
]{charter} 


% El títulos de la memoria, se usa en la carátula y se puede usar el cualquier lugar del documento con el comando \ttitle
\titulo{Implementación de técnicas de visión artificial para la identificación de frutos de duraznero} 

% Nombre del posgrado, se usa en la carátula y se puede usar el cualquier lugar del documento con el comando \degreename
%\posgrado{Carrera de Especialización en Inteligencia Artificial} 
%\posgrado{Carrera de Especialización en Internet de las Cosas} 
\posgrado{Carrera de Especialización en Inteligencia Artificial}
%\posgrado{Maestría en Sistemas Embebidos} 
%\posgrado{Maestría en Internet de las cosas}
% IMPORTANTE: no omitir titulaciones ni tildación en los nombres, también se recomienda escribir los nombres completos (tal cual los tienen en su documento)
% Tu nombre, se puede usar el cualquier lugar del documento con el comando \authorname
\autor{Ing. Sergio Hinojosa}

% El nombre del director y co-director, se puede usar el cualquier lugar del documento con el comando \supname y \cosupname y \pertesupname y \pertecosupname
\director{Título y Nombre del director}
\pertenenciaDirector{pertenencia} 
\codirector{} % para que aparezca en la portada se debe descomentar la opción codirector en los parámetros de documentclass
\pertenenciaCoDirector{}

% Nombre del cliente, quien va a aprobar los resultados del proyecto, se puede usar con el comando \clientename y \empclientename
\cliente{Dr. Gerardo Sánchez}
\empresaCliente{INTA}
 
\fechaINICIO{18 de Junio de 2024}		%Fecha de inicio de la cursada de GdP \fechaInicioName
\fechaFINALPlan{13 de Agosto de 2024} 	%Fecha de final de cursada de GdP
\fechaFINALTrabajo{- de - de 2024}	%Fecha de defensa pública del trabajo final


\begin{document}

\maketitle
\thispagestyle{empty}
\pagebreak


\thispagestyle{empty}
{\setlength{\parskip}{0pt}
\tableofcontents{}
}
\pagebreak


\section*{Registros de cambios}
\label{sec:registro}


\begin{table}[ht]
\label{tab:registro}
\centering
\begin{tabularx}{\linewidth}{@{}|c|X|c|@{}}
\hline
\rowcolor[HTML]{C0C0C0} 
Revisión & \multicolumn{1}{c|}{\cellcolor[HTML]{C0C0C0}Detalles de los cambios realizados} & Fecha      \\ \hline
0      & Creación del documento                                 &\fechaInicioName \\ \hline
1      & Se completa hasta el punto 5 inclusive                & {9} de {Julio} de 2024 \\ \hline
%2      & Se completa hasta el punto 9 inclusive
%		  Se puede agregar algo más \newline
%		  En distintas líneas \newline
%		  Así                                                    & {día} de {mes} de 202X \\ \hline
%3      & Se completa hasta el punto 12 inclusive                & {día} de {mes} de 202X \\ \hline
%4      & Se completa el plan	                                 & {día} de {mes} de 202X \\ \hline

% Si hay más correcciones pasada la versión 4 también se deben especificar acá

\end{tabularx}
\end{table}

\pagebreak



\section*{Acta de constitución del proyecto}
\label{sec:acta}

\begin{flushright}
Buenos Aires, \fechaInicioName
\end{flushright}

\vspace{2cm}

Por medio de la presente se acuerda con \authorname\hspace{1px} que su Trabajo Final de la \degreename\hspace{1px} se titulará ``\ttitle'' y consistirá en la implementación de un algoritmo que permita identificar y cuantificar frutos de duraznos a partir de imágenes de árboles tomadas a campo. El trabajo tendrá un presupuesto preliminar estimado de 600 horas,
% y un costo estimado de \textcolor{red}{\$ XXX}
con fecha de inicio el \fechaInicioName\hspace{1px} y fecha de presentación pública en el mes de Diciembre de 2024.%\fechaFinalName.

Se adjunta a esta acta la planificación inicial.

\vfill

% Esta parte se construye sola con la información que hayan cargado en el preámbulo del documento y no debe modificarla
\begin{table}[ht]
\centering
\begin{tabular}{ccc}
\begin{tabular}[c]{@{}c@{}}Dr. Ing. Ariel Lutenberg \\ Director posgrado FIUBA\end{tabular} & \hspace{2cm} & \begin{tabular}[c]{@{}c@{}}\clientename \\ \empclientename \end{tabular} \vspace{2.5cm} \\ 
\multicolumn{3}{c}{\begin{tabular}[c]{@{}c@{}} \supname \\ Director del Trabajo Final\end{tabular}} \vspace{2.5cm} \\
\end{tabular}
\end{table}



\newpage
\section{1. Descripción técnica-conceptual del proyecto a realizar}
\label{sec:descripcion}
\subsection{Contexto de la implementación}

Contar de forma manual la cantidad frutos que posee un árbol en una plantación agrícola resulta una tarea laboriosa, lenta y propensa a errores. No obstante, conocer estos datos tiene diversas aplicaciones importantes que, debido a su dificultad, no están siendo abordadas. Para un productor, por ejemplo, saber la cantidad de frutos que posee una muestra de su lote durante el raleo\footnote{Descarte de una parte de los frutos para que los restantes tomen más nutrientes de la planta y puedan alcanzar tamaño comercial} le permite calcular la intensidad a aplicar. Además, al momento de la cosecha, le permite tener una estimación de su producción.

\subsection{Objetivo del proyecto}

Este proyecto busca proporcionar al productor:
\begin{itemize}
	\item Precisión y eficiencia: conteo de manera precisa y rápida.
	\item Ahorro de tiempo y costos: reducción del tiempo y de costos asociados con la mano de obra requerida para esta tarea.
	\item Optimización del rendimiento agrícola: permitir toma de decisiones informadas sobre la gestión del cultivo, la cosecha y la planificación de la mano de obra.
	\item Monitoreo temprano de la producción.
\end{itemize}

Por otro lado, los algoritmos a desarrollar resultan una herramienta muy útil, ya que permiten la obtención de un gran volumen de datos que pueden vincularse con diferentes características de interés, tales como:

\begin{itemize}
	\item Porcentaje de cuajado: relacionado con la producción.
	\item Potencial de raleo: capacidad de genotipo a soportar mayor o menor raleo.
	\item Rendimiento.
\end{itemize}

A esto se lo denomina fenotipo y esta herramienta fenómica será fundamental para
alimentar modelos de IA que combinan datos genómicos con variables ambientales.

El siguiente diagrama de bloques ilustra el pipeline del sistema, los diferentes subsistemas involucrados y el flujo de datos:

\begin{figure}[htpb]
\centering 
\includegraphics[width=.85\textwidth]{./Figuras/diagrama.png}
\caption{Diagrama en bloques del sistema.}
\label{fig:diagBloques}
\end{figure}

\vspace{25px}



\section{2. Identificación y análisis de los interesados}
\label{sec:interesados}

\begin{table}[ht]
%\caption{Identificación de los interesados}
%\label{tab:interesados}
\begin{tabularx}{\linewidth}{@{}|l|X|X|l|@{}}
\hline
\rowcolor[HTML]{C0C0C0} 
Rol           & Nombre y Apellido & Organización 	& Puesto 	\\ \hline
%Auspiciante   &                   &              	&        	\\ \hline
Cliente       & \clientename      &\empclientename	& Director Científico Biotango Technologies SAS\\ \hline
%Impulsor      &                   &              	&        	\\ \hline
Responsable   & \authorname       & FIUBA        	& Alumno 	\\ \hline
%Colaboradores &                   &              	&        	\\ \hline
Orientador    & \supname	      & \pertesupname 	& Director del Trabajo Final \\ \hline
%Equipo        & miembro1 \newline 
%				miembro2          &              	&        	\\ \hline
%Opositores    &                   &              	&        	\\ \hline
%Usuario final &                   &              	&        	\\ \hline

Cliente: El Dr. Gerardo Sanchez tiene más de 20 años de experiencia en biotecnología frutícola, colaborará en los requerimientos y en la válidación de los modelos desarrollados.

\end{tabularx}
\end{table}

\section{3. Propósito del proyecto}
\label{sec:proposito}

El propósito de este proyecto es desarrollar un sistema informático capaz de detectar y contar los frutos de un duraznero en el árbol a partir de imágenes tomadas por una cámara 2D.

\section{4. Alcance del proyecto}
\label{sec:alcance}


El proyecto incluye:
\begin{itemize}
	\item Procesamiento de la imagen tomada por un celular.
	\item Detección del árbol a analizar.
	\item Conteo de los frutos.
\end{itemize}

El proyecto no incluye:
\begin{itemize}
	\item Procesamiento de imágenes en blano y negro.
	\item Procesamiento de video.
	\item Implementación industrial. No se realizará un deploy en ningún sistema embebido ni en la nube, el sistema semantendrá en un entorno de desarrollo.
\end{itemize}


\section{5. Supuestos del proyecto}
\label{sec:supuestos}

Para el desarrollo del presente proyecto se supone que se contará con:

\begin{itemize}
    \item Horas de trabajo del desarrollador y del director del proyecto.
	\item Un ordenador con placa gráfica dedicada.
	\item Banco de imágenes de campo reales para el entrenamiento del sistema.
	\item No será necesario una inversión en hardware.
	\item No se espera una precisión del 100 \%. Se definirá un margen aceptable.
\end{itemize}

\section{6. Requerimientos}
\label{sec:requerimientos}

\begin{enumerate}
	\item Requerimientos funcionales:
	\begin{enumerate}
		\item El sistema debe ser capaz de identificar y cuantificar frutos tanto en estado inmaduro o maduro a partir de fotos de árboles tomadas en la plantación. 
		\item El sistema debe ser capaz de analizar fotos en formato .jpeg.
		\item El sistema deberá indicar el grado de confianza de la detección y el error esperado en las mediciones.
		\item El sistema debe mostrar los resultados en una interfaz gráfica y debe exportar los resultados a formato .cvs 
		\item Los dataset y código en repositorios deben mantenerse NO abiertos, de modo de mantener la confidencialidad del trabajo.
		\item El usuario debe poder analizar una foto individual o un grupo de fotos, por lo menos grupos de 20 imágenes.
		\item El usuario debe ser capaz de cambiar el nivel de confianza para modificar la detección.
	\end{enumerate}

	\item Requerimientos de documentación:
	\begin{enumerate}
		\item Se debe proveer el códigos base.
		\item Se deben describir los ensayos y condiciones para la reporducción de los mismos.
		\item Se debe generar el informe de avance y la memoria final del trabajo.
		\item Se debe proveer un manual de instrucciones para la instalación y uso.
	\end{enumerate}
	
	\item Requerimientos de testing:
	\begin{enumerate}
		\item El modelo debe ser entrenado a partir de modelos pre-entrenados utilizando la técnica de transfer learning.
		\item El entrenamiento del modelo debe realizarse a partir de un dataset creado con imágenes propias y de terceros respetando los derechos de uso establecidos por los mismos.
		\item El entrenamiento y las pruebas en los datasets de validación y testing deben ser llevadas adelante en notebooks escritas en lenguaje de programación Python 3.8 o superior.
	\end{enumerate}
	
	\item Requerimientos de la interfaz:
	\begin{enumerate}
		\item La intefaz debe ser intuitiva, con capacidad de cargar las fotos de manera sencilla.
	\end{enumerate}

\end{enumerate}

\section{7. Historias de usuarios (\textit{Product backlog})}
\label{sec:backlog}

El criterio establecido para la asignación de los puntajes de las historias de usuario se basa en los criterios:
\begin{enumerate}
	\item Esfuerzo requerido en tiempo.
	\item Complejidad de la tarea.
	\item Riesgo asociado.
\end{enumerate}
Cada uno de los criterios individuales puede tener un valor entre 0 (bajo) y 5 (alto). El puntaje final será la suma de los puntajes individuales y se le asignará el valor más próximo de los valores de la serie de Fibonacci: 1, 2, 3, 5, 8, 13, 20.

\begin{itemize}
\item quiero un sistema de visión artificial que detecte y cuente automáticamente la cantidad de duraznos en mis árboles, para poder monitorear y gestionar mejor mi producción.
	\begin{itemize}
	\item complejidad: 4
	\item dificultad: 3
	\item incertidumbre: 3
	\item \textit{Story points}: 13 
	\end{itemize}
\end{itemize}

\begin{itemize}
\item Como usuario quiero poder capturar imágenes de mis árboles utilizando una cámara y poder subirla al sistema para el conteo de los frutos.
	\begin{itemize}
	\item complejidad: 1
	\item dificultad: 1
	\item incertidumbre: 2
	\item \textit{Story points}: 5 
	\end{itemize}
\end{itemize}

\begin{itemize}
\item Como usuario quiero poder configurar parámetros del sistema (como la sensibilidad de detección), para ajustar la precisión del conteo según las condiciones específicas de mi plantación.
	\begin{itemize}
	\item complejidad: 4
	\item dificultad: 2
	\item incertidumbre: 4
	\item \textit{Story points}: 8 
	\end{itemize}
\end{itemize}

\begin{itemize}
\item Como usuario quiero recibir un reporte con la cantidad de duraznos detectados en cada árbol, para poder tomar decisiones informadas sobre mi cultivo.
	\begin{itemize}
	\item complejidad: 2
	\item dificultad: 2
	\item incertidumbre: 2
	\item \textit{Story points}: 8 
	\end{itemize}
\end{itemize}

\begin{itemize}
\item Como desarrollador, quiero establecer un entorno de desarrollo escalable, bien documentado y organizado para facilitar la mejora continua del sistema.
	\begin{itemize}
	\item complejidad: 3
	\item dificultad: 5
	\item incertidumbre: 3
	\item \textit{Story points}: 13 
	\end{itemize}
\end{itemize}

\begin{itemize}
\item Como desarrollador quiero crear un sistema de notificaciones de error y logueo que alerte sobre los problemas en la detección y sea más sencillo remediar bugs o corregir parámetros de configuración.
	\begin{itemize}
	\item complejidad: 3
	\item dificultad: 3
	\item incertidumbre: 2
	\item \textit{Story points}: 8 
	\end{itemize}
\end{itemize}

\begin{itemize}
\item Como desarrollador quiero diseñar una base de datos para almacenar las imágenes y los resultados de detección, para poder acceder y analizar los datos históricos.
	\begin{itemize}
	\item complejidad: 2
	\item dificultad: 2
	\item incertidumbre: 2
	\item \textit{Story points}: 8 
	\end{itemize}
\end{itemize}

\section{8. Entregables principales del proyecto}
\label{sec:entregables}

Los entregables del proyecto son:

\begin{itemize}
	\item Documento de arquitecutra de software.
	\item Notebooks desarrolladas para el entrenamiento del modelo.
	\item Datasets utilizados.
	\item Manual de usuario.
	\item Informe de avance.
	\item Informe final.
\end{itemize}

\section{9. Desglose del trabajo en tareas}
\label{sec:wbs}

\begin{enumerate}
\item Planificación. (50 hs)
	\begin{enumerate}
	\item Estudio de necesidades. (10 hs)
	\item Análisis de requerimientos. (10 hs)
	\item Análisis de factibilidad. (10 hs)
	\item Confección del documento de planificación. (20 hs)
	\end{enumerate}
	
\item Investigación preliminar. (110 hs)
	\begin{enumerate}
	\item Investigación de modelos de detección y clasificación de objetos. (40 hs)
	\item Desarrollo de tests de performance para la aplicación. (40 hs)
	\item Análisis de los resultados y selección del modelo final. (30 hs)
	\end{enumerate}
	
\item Recopilación de imágenes para los datasets. (60 hs)
	\begin{enumerate}
	\item Búsqueda de datasets de terceros. (20 hs)
	\item Captura de imágenes. (30 hs)
	\item Preparación del dataset. (20 hs)
	\item Refinamiento y seleccción final. (10 hs)
	\end{enumerate}

\item Entrenamiento de la red neuronal. (120 hs)
	\begin{enumerate}
	\item Desarrollo del modelo. (40hs)
	\item Entrenamiento del modelo. (40hs)
	\item Ajuste de hiperparámetros. (30hs)
	\item Refinamiento del dataset. (10hs)
	\end{enumerate}
	
\item Desarrollo del sistema global. (120 hs)
	\begin{enumerate}
	\item Desarrollo de una interfaz de comandos de alto nivel. (40 hs)
	\item Desarrollo del sistema de registro en base de datos. (20 hs)
	\item Evaluación de desempeño del sistema. (20 hs)
	\item Desarrollo de demos. (40 hs)
	\end{enumerate}
	
\item Elaboración de documentación y presentación. (60 hs)
	\begin{enumerate}
	\item Elaborar el informe final del proyecto. (40 hs)
	\item Entrenamiento del modelo. (30hs)
	\item Preparar presentación final. (20 hs)
	\end{enumerate}
\end{enumerate}

Cantidad total de horas: 520 hs.

\section{10. Diagrama de Activity On Node}
\label{sec:AoN}

\begin{consigna}{red}
Armar el AoN a partir del WBS definido en la etapa anterior.

Una herramienta simple para desarrollar los diagramas es el Draw.io (\url{https://app.diagrams.net/}).
\href{https://app.diagrams.net}{Draw.io}


\begin{figure}[htpb]
\centering 
\includegraphics[width=.8\textwidth]{./Figuras/AoN.png}
\caption{Diagrama de \textit{Activity on Node}.}
\label{fig:AoN}
\end{figure}

Indicar claramente en qué unidades están expresados los tiempos.
De ser necesario indicar los caminos semi críticos y analizar sus tiempos mediante un cuadro.
Es recomendable usar colores y un cuadro indicativo describiendo qué representa cada color.

\end{consigna}

\section{11. Diagrama de Gantt}
\label{sec:gantt}

\begin{consigna}{red}
Existen muchos programas y recursos \textit{online} para hacer diagramas de Gantt, entre los cuales destacamos:

\begin{itemize}
\item Planner
\item GanttProject
\item Trello + \textit{plugins}. En el siguiente link hay un tutorial oficial: \\ \url{https://blog.trello.com/es/diagrama-de-gantt-de-un-proyecto}
\item Creately, herramienta online colaborativa. \\\url{https://creately.com/diagram/example/ieb3p3ml/LaTeX}
\item Se puede hacer en latex con el paquete \textit{pgfgantt}\\ \url{http://ctan.dcc.uchile.cl/graphics/pgf/contrib/pgfgantt/pgfgantt.pdf}
\end{itemize}

Pegar acá una captura de pantalla del diagrama de Gantt, cuidando que la letra sea suficientemente grande como para ser legible. 
Si el diagrama queda demasiado ancho, se puede pegar primero la ``tabla'' del Gantt y luego pegar la parte del diagrama de barras del diagrama de Gantt.

Configurar el software para que en la parte de la tabla muestre los códigos del EDT (WBS).\\
Configurar el software para que al lado de cada barra muestre el nombre de cada tarea.\\
Revisar que la fecha de finalización coincida con lo indicado en el Acta Constitutiva.

En la figura \ref{fig:gantt}, se muestra un ejemplo de diagrama de gantt realizado con el paquete de \textit{pgfgantt}. 
En la plantilla pueden ver el código que lo genera y usarlo de base para construir el propio.

Las fechas pueden ser calculadas utilizando alguna de las herramientas antes citadas. Sin embargo, el siguiente ejemplo
fue elaborado utilizando 
\href{https://docs.google.com/spreadsheets/d/1fBz8NhSpc4tkkhz3KjJCbh1nR_ltDkfEcZi4tZXduqs}{esta hoja de cálculo}.

Es importante destacar que el ancho del diagrama estará dado por la longitud del texto utilizado para las tareas 
(Ejemplo: tarea 1, tarea 2, etcétera) y el valor \textit{x unit}. Para mejorar la apariencia del diagrama, es necesario
ajustar este valor y, quizás, acortar los nombres de las tareas.

\begin{figure}[htpb]
  \begin{center}
    \begin{ganttchart}[
      time slot unit=day,
      time slot format=isodate,
      x unit=0.038cm,
      y unit title=0.7cm,
      y unit chart=0.6cm,
      milestone/.append style={xscale=4}
      ]{2021-03-05}{2021-12-16}
      \gantttitlecalendar*{2021-03-05}{2021-12-16}{year} \\
      \gantttitlecalendar*{2021-03-05}{2021-12-16}{month} \\
      \ganttgroup{Duración Total}{2021-03-05}{2021-12-16} \\
      %%%%%%%%%%%%%%%%%Organización
      \ganttgroup{Organización}{2021-03-05}{2021-04-16} \\
      \ganttbar{Planificación del proyecto}{2021-03-05}{2021-04-15} \\
      %%%%%%%%%%%%%%%%%Ejecución
      \ganttgroup{Ejecución}{2021-04-16}{2021-10-21} \\
      \ganttbar{Tarea 1}{2021-04-16}{2021-04-29} \\
      \ganttbar{Tarea 2}{2021-04-30}{2021-05-13} \\
      \ganttbar{Tarea 3}{2021-05-14}{2021-05-27} \\
      \ganttbar{Tarea 4}{2021-05-28}{2021-07-12} \\
      \ganttbar{Tarea 5}{2021-07-13}{2021-08-09} \\
      \ganttbar{Tarea 6}{2021-08-10}{2021-09-23} \\
      \ganttbar{Tarea 7}{2021-09-24}{2021-09-30} \\
      \ganttbar{Tarea 8}{2021-10-01}{2021-10-14} \\
      \ganttbar{Tarea 9}{2021-10-15}{2021-10-21} \\
      % %%%%%%%%%%%%%%%%%Finalización
      \ganttgroup{Finalización}{2021-10-22}{2021-12-16} \\
      \ganttbar{Memoria v1}{2021-10-22}{2021-11-04} \\
      \ganttbar{Memoria v2}{2021-11-05}{2021-11-18} \\
      \ganttbar{Memoria final}{2021-11-19}{2021-12-02} \\
      % La fecha del siguiente milestone es la fecha en que terminamos la memoria
      \ganttmilestone{Enviar memoria al director}{2021-12-02} \\
      \ganttbar{Elaborar la presentación}{2021-12-03}{2021-12-16} \\
      \ganttmilestone{Ensayo de la presentación}{2021-12-16} \\
      %%%%%%%%%%%%%%%%%%%%%%%%%%%%%%%%%%%%%%%%%%%%%%%%%%%%%%%%%%%%%%%
    \end{ganttchart}
  \end{center}
  \caption{Diagrama de gantt de ejemplo}
  \label{fig:gantt}
\end{figure}


\begin{landscape}
\begin{figure}[htpb]
\centering 
\includegraphics[height=.85\textheight]{./Figuras/Gantt-2.png}
\caption{Ejemplo de diagrama de Gantt (apaisado).} %Modificar este título acorde.
\label{fig:diagGantt}
\end{figure}

\end{landscape}

\end{consigna}


\section{12. Presupuesto detallado del proyecto}
\label{sec:presupuesto}

\begin{consigna}{red}
Si el proyecto es complejo entonces separarlo en partes:
\begin{itemize}
	\item Un total global, indicando el subtotal acumulado por cada una de las áreas.
	\item El desglose detallado del subtotal de cada una de las áreas.
\end{itemize}

IMPORTANTE: No olvidarse de considerar los COSTOS INDIRECTOS.

Incluir la aclaración de si se emplea como moneda el peso argentino (ARS) o si se usa moneda extranjera (USD, EUR, etc). Si es en moneda extranjera se debe indicar la tasa de conversión respecto a la moneda local en una fecha dada.

\end{consigna}

\begin{table}[htpb]
\centering
\begin{tabularx}{\linewidth}{@{}|X|c|r|r|@{}}
\hline
\rowcolor[HTML]{C0C0C0} 
\multicolumn{4}{|c|}{\cellcolor[HTML]{C0C0C0}COSTOS DIRECTOS} \\ \hline
\rowcolor[HTML]{C0C0C0} 
Descripción &
  \multicolumn{1}{c|}{\cellcolor[HTML]{C0C0C0}Cantidad} &
  \multicolumn{1}{c|}{\cellcolor[HTML]{C0C0C0}Valor unitario} &
  \multicolumn{1}{c|}{\cellcolor[HTML]{C0C0C0}Valor total} \\ \hline
 &
  \multicolumn{1}{c|}{} &
  \multicolumn{1}{c|}{} &
  \multicolumn{1}{c|}{} \\ \hline
 &
  \multicolumn{1}{c|}{} &
  \multicolumn{1}{c|}{} &
  \multicolumn{1}{c|}{} \\ \hline
\multicolumn{1}{|l|}{} &
   &
   &
   \\ \hline
\multicolumn{1}{|l|}{} &
   &
   &
   \\ \hline
\multicolumn{3}{|c|}{SUBTOTAL} &
  \multicolumn{1}{c|}{} \\ \hline
\rowcolor[HTML]{C0C0C0} 
\multicolumn{4}{|c|}{\cellcolor[HTML]{C0C0C0}COSTOS INDIRECTOS} \\ \hline
\rowcolor[HTML]{C0C0C0} 
Descripción &
  \multicolumn{1}{c|}{\cellcolor[HTML]{C0C0C0}Cantidad} &
  \multicolumn{1}{c|}{\cellcolor[HTML]{C0C0C0}Valor unitario} &
  \multicolumn{1}{c|}{\cellcolor[HTML]{C0C0C0}Valor total} \\ \hline
\multicolumn{1}{|l|}{} &
   &
   &
   \\ \hline
\multicolumn{1}{|l|}{} &
   &
   &
   \\ \hline
\multicolumn{1}{|l|}{} &
   &
   &
   \\ \hline
\multicolumn{3}{|c|}{SUBTOTAL} &
  \multicolumn{1}{c|}{} \\ \hline
\rowcolor[HTML]{C0C0C0}
\multicolumn{3}{|c|}{TOTAL} &
   \\ \hline
\end{tabularx}%
\end{table}


\section{13. Gestión de riesgos}
\label{sec:riesgos}

\begin{consigna}{red}
a) Identificación de los riesgos (al menos cinco) y estimación de sus consecuencias:
 
Riesgo 1: detallar el riesgo (riesgo es algo que si ocurre altera los planes previstos de forma negativa)
\begin{itemize}
	\item Severidad (S): mientras más severo, más alto es el número (usar números del 1 al 10).\\
	Justificar el motivo por el cual se asigna determinado número de severidad (S).
	\item Probabilidad de ocurrencia (O): mientras más probable, más alto es el número (usar del 1 al 10).\\
	Justificar el motivo por el cual se asigna determinado número de (O). 
\end{itemize}   

Riesgo 2:
\begin{itemize}
	\item Severidad (S): X.\\
	Justificación...
	\item Ocurrencia (O): Y.\\
	Justificación...
\end{itemize}

Riesgo 3:
\begin{itemize}
	\item Severidad (S):  X.\\
	Justificación...
	\item Ocurrencia (O): Y.\\
	Justificación...
\end{itemize}


b) Tabla de gestión de riesgos:      (El RPN se calcula como RPN=SxO)

\begin{table}[htpb]
\centering
\begin{tabularx}{\linewidth}{@{}|X|c|c|c|c|c|c|@{}}
\hline
\rowcolor[HTML]{C0C0C0} 
Riesgo & S & O & RPN & S* & O* & RPN* \\ \hline
       &   &   &     &    &    &      \\ \hline
       &   &   &     &    &    &      \\ \hline
       &   &   &     &    &    &      \\ \hline
       &   &   &     &    &    &      \\ \hline
       &   &   &     &    &    &      \\ \hline
\end{tabularx}%
\end{table}

Criterio adoptado: 

Se tomarán medidas de mitigación en los riesgos cuyos números de RPN sean mayores a...

Nota: los valores marcados con (*) en la tabla corresponden luego de haber aplicado la mitigación.

c) Plan de mitigación de los riesgos que originalmente excedían el RPN máximo establecido:
 
Riesgo 1: plan de mitigación (si por el RPN fuera necesario elaborar un plan de mitigación).
  Nueva asignación de S y O, con su respectiva justificación:
  \begin{itemize}
	\item Severidad (S*): mientras más severo, más alto es el número (usar números del 1 al 10).
          Justificar el motivo por el cual se asigna determinado número de severidad (S).
	\item Probabilidad de ocurrencia (O*): mientras más probable, más alto es el número (usar del 1 al 10).
          Justificar el motivo por el cual se asigna determinado número de (O).
	\end{itemize}

Riesgo 2: plan de mitigación (si por el RPN fuera necesario elaborar un plan de mitigación).
 
Riesgo 3: plan de mitigación (si por el RPN fuera necesario elaborar un plan de mitigación).

\end{consigna}


\section{14. Gestión de la calidad}
\label{sec:calidad}

\begin{consigna}{red}
Elija al menos diez requerimientos que a su criterio sean los más importantes/críticos/que aportan más valor y para cada uno de ellos indique las acciones de verificación y validación que permitan asegurar su cumplimiento.

\begin{itemize} 
\item Req \#1: copiar acá el requerimiento con su correspondiente número.

\begin{itemize}
	\item Verificación para confirmar si se cumplió con lo requerido antes de mostrar el sistema al cliente. Detallar.
	\item Validación con el cliente para confirmar que está de acuerdo en que se cumplió con lo requerido. Detallar. 
\end{itemize}

\end{itemize}

Tener en cuenta que en este contexto se pueden mencionar simulaciones, cálculos, revisión de hojas de datos, consulta con expertos, mediciones, etc.  

Las acciones de verificación suelen considerar al entregable como ``caja blanca'', es decir se conoce en profundidad su funcionamiento interno.  

En cambio, las acciones de validación suelen considerar al entregable como ``caja negra'', es decir, que no se conocen los detalles de su funcionamiento interno.

\end{consigna}

\section{15. Procesos de cierre}    
\label{sec:cierre}

\begin{consigna}{red}
Establecer las pautas de trabajo para realizar una reunión final de evaluación del proyecto, tal que contemple las siguientes actividades:

\begin{itemize}
	\item Pautas de trabajo que se seguirán para analizar si se respetó el Plan de Proyecto original:\\
	 - Indicar quién se ocupará de hacer esto y cuál será el procedimiento a aplicar. 
	\item Identificación de las técnicas y procedimientos útiles e inútiles que se emplearon, los problemas que surgieron y cómo se solucionaron:\\
	 - Indicar quién se ocupará de hacer esto y cuál será el procedimiento para dejar registro.
	\item Indicar quién organizará el acto de agradecimiento a todos los interesados, y en especial al equipo de trabajo y colaboradores:\\
	  - Indicar esto y quién financiará los gastos correspondientes.
\end{itemize}

\end{consigna}

\end{document}